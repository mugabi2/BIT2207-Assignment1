\documentclass[12pt]{article}
\usepackage[left=33mm,top=10mm,bottom=15mm]{geometry}
\begin{document}
\title{RESEARCH ON HOW TO CUT GLASS FAST AND NEATLY\\}
\maketitle
BY MUGABI SAMUEL 15/U/7793/PS 215007759
\section{Problem}
At first the problem is knowing how cut the glass bottle before knowing how to cut it fast and neatly as stated above. 
\section{Scope}
In this research, a better approach to glass cutting is developed. The approach is neat and fast.
\section{Interview}
So I conducted an interview asking people how to cut glass bottles. Different views were raised which included using a sharp knife, hitting the bottle with a stone, using a saw blade, using a glass cutter.
\section{Internet methods}
I then continued my research to the internet where I looked for tutorials on how to cut glass bottle. I watched different tutorials which showed different ways of cutting glass. These included;
Using a glass cutter to mark of where you want your cut to pass. The glass cutter is held in the hand and then first dipped into oil or paraffin. Then glass cutter is then pressed on the bottle marking off where you want the cut to pass. This is followed by continuous heating of the marked area with candle fire. The fire should be steady hence only heating the marked area of the bottle. This is followed by immediate cooling using either water or ice. The heating and cooling of the marked area is continued until the bottle snaps along the marked area.
Using the glass cutter to mark of where you want the cut to be. The same procedure, like in method one, is repeated though this time the glass cutter is fixed on a bench. The bottle is also fixed using two wooden blocks but allowed to rotate while touching the glass cutter. The procedure is commenced by rotating the glass bottle on the fixed glass cutter. A mark is achieved. Continuous heating and cooling, like in method one, are is carried out.
Another method was to dip a string of sisal or cotton in paraffin, petrol or oil. The string is then wrapped around the bottle where the cut is to pass. This method should be carried out with a lot of care since the petrol which is highly flammable may cause injuries. The string is then lit with a match stick. The bottle is rotated horizontally allowing the heat to go around the bottle where the cut is to pass. The bottle is heated until one feels that the bottle has heated up enough. The bottle is immediately immersed in water allowing it to cool instantaneously. The heating and cooling is carried out until the bottle snaps along the string marked area.
Using masking tape mark off where the dremel tool is to cut. The dremel tool is fixed in one position. The bottle is then rotated along the cutting dremel.

\section{Materials required}
I needed to get the required material for my research. These included used wine bottles, a glass cutter, sisal rope, candle, water, ice, a bench, two blocks of wood, paraffin, a match box, basin.
\section{Experimenting}
In my current research I needed to know which method is faster and does the cut neater. So I tried out a number of methods. Methods from one up to three were carried out and all were successful though the better method was the one that resulted a neat circular cut on the bottle and faster than others.
\section{Challenges}
As any other research, challenges were faced. These included;
The method of using petrol or paraffin was too risky since it could have caused burns and injuries.
Lack of the dremel machine to carry out the fourth method.
The glass cutter lost its sharpness.
Some methods took a long time before the bottled snapped.

\section{Faster method}
Out of the methods carried out, I gained experience in glass bottle cutting and therefore made my own method which is faster than ones stated above. This method requires one to first mark off the cut area using a glass cutter which was dipped in oil. The marked area is then cooled using ice. This is done until the marked area is as cool as possible. The bottle mark is then heated immediately using a candle flame. Normally, if the cooling was good enough, the bottle snaps during the heating. Else one can go ahead and pour water on the mark. For the worst case, using this method, the snapping shouldn’t go beyond pouring water. A neat cut is obtained faster.
\section{Conclusion}
The faster and neater method was that I derived out of experience and it is advisable to use this since it is safer.
\end{document}
